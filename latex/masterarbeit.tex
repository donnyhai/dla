\documentclass[12pt,a4paper]{scrartcl}

\usepackage[utf8]{inputenc}
\usepackage[T1]{fontenc}
\usepackage[british,UKenglish,USenglish,american]{babel}

\usepackage{bbm}
\usepackage{hyperref}
\usepackage[pdftex]{graphicx}
\usepackage{graphicx}
\usepackage{latexsym}
\usepackage{amsmath,amssymb,amsthm}
\allowdisplaybreaks
\usepackage{dsfont}
\usepackage{pifont}
\usepackage{nicefrac}
\usepackage{textcomp}
\usepackage{enumitem}
\usepackage{lmodern}


% Abstand obere Blattkante zur Kopfzeile ist 2.54cm - 15mm
\setlength{\topmargin}{-15mm}	   
                  
%\numberwithin{equation}{section} 
\numberwithin{equation}{subsection}

\newcommand{\C}{\mathbb{C}} % komplexe
\newcommand{\R}{\mathbb{R}} % reelle
\newcommand{\Q}{\mathbb{Q}} % rationale
\newcommand{\Z}{\mathbb{Z}} % ganze
\newcommand{\N}{\mathbb{N}} % natuerliche
\newcommand{\PP}{\mathbb{P}} % Probability
\newcommand{\E}{\mathcal{E}} % big Epsilon
\newcommand{\EE}{\mathbb{E}} %Expectation value
\newcommand{\K}{\mathcal{K}}
\newcommand{\1}{\mathbbm{1}}
\newcommand{\G}{\mathcal{G}}
\newcommand{\GG}{\mathfrak{G}}
\newcommand{\mP}{\mathcal{P}}
\newcommand{\rad}{\text{rad}}

\renewcommand{\labelenumi}{(\roman{enumi})}

\numberwithin{equation}{section}

\theoremstyle{definition}
\newtheorem{example}{Example}[subsection]
\newtheorem{theorem}{Theorem}[subsection]
\newtheorem{corollary}{Corollary}[subsection]
\newtheorem{lemma}{Lemma}[subsection]
\newtheorem{definition}{Definition}[subsection]
\newtheorem{proposition}{Proposition}[subsection]
\newtheorem{algorithm}{Algorithm}[subsection]
\newtheorem{prop}{Proposition}[subsection]
\newtheorem{remark}{Remark}[subsection]
\newtheorem{pro}{Proof}
\newtheorem{comment}{Comment}[subsection]


\begin{document}
	\pagestyle{empty}

\begin{titlepage}

	\includegraphics[scale=0.45]{kit-logo.jpg} 
    \vspace*{2cm} 
\begin{center} \large 
    
   Masterthesis
    \vspace*{2cm}

    {\huge External DLA}\\
    \vspace*{2.5cm}

    Tillmann Tristan Bosch
    \vspace*{1.5cm}

    10. March 2020
    \vspace*{3.5cm}


    Supervisor: PD. Dr. Steffen Winter \\[1cm]
    Faculty of Mathematics\\[1cm]
	Karlsruhe Institute of Technology
\end{center}
\end{titlepage}

\newpage

\newpage
\phantom \\
\newpage

\tableofcontents %Inhaltsverzeichnis

 	\pagestyle{headings}

\setcounter{page}{1}


\newpage

\section{Introduction}
\includegraphics[scale=0.04]{display.jpg} 
\includegraphics[scale=0.091]{display2.jpg}





\newpage


\section{Preliminaries} \label{prelim}

\subsection{Symbols}
Here we list all symbols which will be used in this paper. Let $d\in \N$ and $q\in \{0,\dots,d\}$. 
\begin{flalign*}
	&\N = \{1,2,3,\dots\},\quad \text{the set of natural numbers (without 0)}\\
	&\N_0 = \N\cup\{0\}\\
	&\N^\infty = \N \cup \{\infty\}\\
	&\mathcal{B}^d,\quad \text{$d$-dimensional Borel-$\sigma$-algebra of $\R^d$} \\
	&\K^d,\quad \text{the set of convex and compact sets in }\R^d\\
	&B_d(x,r) = \{y\in \R^d\ |\ |x-y| \leq r\},\quad  \text{the $d$-dimensional closed ball of radius $r$ around $x$}\\
	&B_r := B_2(r,0) \\
	&S_{d-1}(r,x) = \partial B_d(r,x),\quad  \text{the $(d-1)$-dimensional surface of the $d$-dimensional ball}\\
	&A(d,q),\quad \text{the set of q-dimensional affine subspaces of }\R^d \\
	&\mathcal{A}(d,q),\quad  \text{the $\sigma$-algebra of } A(d,q), \text{ as constructed later in the paper} \\
	&\G := A(2,1),\quad  \text{the set of lines in the real plane}\\
	&SO_d := \{\nu \in \R^{d\times d}\ |\ \nu \nu^\top = I_d \text{ and } \det \nu = 1\},\quad \text{identify }SO_2 = \{\nu_\beta := e^{i\beta} \in \C\ |\ \beta\in [0,2\pi)\} \\
	&G_d := \{\varphi: \R^d \to \R^d, x\mapsto \nu x+b\ |\ \nu \in SO_d,b\in \R^d\},\quad  \text{the set of euclidean motions} \\
	&\mathcal{P}_f,\quad \text{the set of finite subsets of $\Z^d$ (or $\Z^2$ respective to the context)} \\
\end{flalign*}

\newpage
\subsection{Basic structures}

\subsubsection{Graphs}

Let $d\in \N$. We will be interested in the undirected graph $(\mathbb{Z}^d, E)$ with its canonical graph structure, which is two vertices (or points) $x=(x_1,\dots,x_d),y=(y_1,\dots,y_d)\in \mathbb{Z}^d$ form an edge (e.q. $\{x,y\}\in E$) if and only if there exists exactly one $i\in \{1,\dots, d\}$ such that $|x_i - y_i| = 1$ and $x_j = y_j$ for all $j\neq i$. For a point $x\in \mathbb{Z}^d$ its set of $\mathit{neighbours}$ is defined as 
\begin{align*}
	N(x) := \{y\in \mathbb{Z}^d\ |\ \{x,y\}\in E\}.
\end{align*}
For a set $A\subset \mathbb{Z}^d$ the $\mathit{outer\ boundary}\ \partial A$ of $A$ is defined as 
\begin{align*}
	\partial A := \{y\in \mathbb{Z}^d\setminus A\ |\ \exists x\in A:\ \{x,y\}	\in E\}
\end{align*}
Instead of $(\mathbb{Z}^d, E)$ we will write $\mathbb{Z}^d$ from now on. 

\subsubsection{Random Walks}

\begin{definition}
	Throughout the paper let  $(\Omega,\mathcal{F}, \mathbb{P})$ be a probability space and for our space of interest $\mathbb{Z}^d$ we will always use the discrete $\sigma$-algebra which is the power set of $\mathbb{Z}^d$. A family $(S_n)_{n\in \mathbb{N}_0}$ of measurable functions $S_n: \Omega \to \mathbb{Z}^d$ is called a $\mathit{random\ walk\ on}\ \mathbb{Z}^d$ $\mathit{(starting\ at}\ x\in \mathbb{Z}^d)$ if and only if $S_0=x$ a.s. and 
	
	\begin{align*}
		\mathbb{P}(S_n = y\ |\ S_{n-1} = z) = \frac{1}{|N(z)|} = \frac{1}{2d},\quad \text{ for all }  y\in N(z) \text{ and } z\in \Z^d.
	\end{align*}
	
	\noindent Note that $|N(z)| = 2d$ for all $z\in \mathbb{Z}^d$ since every point has two neighbours in the direction of every component. We can therefore conclude easily that $\mathbb{P}(S_n = y\ |\ S_{n-1} = z) = 0$ for all $y\notin N(z)$ and $z\in \Z^d$. For $x,y\in\Z^d$ we introduce the short notation
	\begin{flalign*}
		\PP_x(S_n=y) := \PP(S_n = y\ |\ S_0 = x). 
	\end{flalign*}
	
\end{definition}
So a random walk can be understood as a particle starting from some point $x$ and moving randomly on the grid choosing its next step uniformly from its neighbours. For the following let $(S_n)_{n\in \mathbb{N}}$ be a random walk on $\Z^d$ starting at $x\in \Z^d$. 

\begin{definition}
	Let $A\subset \Z^d$. We define the $hitting\ times$ of A by
	
	\begin{align*}
		T_A := \min \{n\geq 0\ |\ S_n\in A\}\text{ and } T^+_A := \min \{n\geq 1\ |\ S_n\in A\}, 
	\end{align*}
	
	\noindent and $T_y:= T_{\{y\}}$ and $T^+_y:= T^+_{\{y\}}$ for $y\in \Z^d$.
\end{definition}

\begin{definition}
	Define $T:=T^+_x$ to be the first time for the random walk to come back to its origin $x$. The random walk is called $\mathit{recurrent}$ if 
	\begin{flalign*}
		\PP_x(T<\infty) = 1, 
	\end{flalign*}
	and $\mathit{transient}$ if
	\begin{flalign*}
		\PP_x(T<\infty) < 1.
	\end{flalign*}
\end{definition}

\begin{lemma} \label{recurr}
	A random walk $(S_n)_{n\in \N}$ on $\Z^d$ is recurrent if $d\leq 2$ and transient if $d\geq 3$. 
\end{lemma}
\begin{proof}
	A detailed proof of this result is presented in $\cite{henze}$ Satz $5.1$. 
\end{proof}

\subsubsection{Expressions}

If for $A\in \mathcal{F}$ we have $\mathbb{P}(A)=1$ we will say that $\glqq A \text{ holds }\mathbb{P}\text{-a.s.}\grqq$, or short $\glqq A\text{ holds } \text{a.s.}\grqq$ ($A$ holds almost surely). Another short expression, if for a set of logical statements $(A_t)_{t\in I}$ with $I\in\{\N,\R\}$ we say $\glqq$$A_t$ holds for large $t$$\grqq$ it shall mean that there exists $T\in I$ such that $A_t$ holds for all $t>T$. Here we mean that a logical statement holds if and only if the statement is true. 


\newpage
\section{Incremental Aggregate}

\subsection{Definition}

In this paper we will look at stochastic processes on the set of finite subsets of $\mathbb{Z}^d$, where we start with the one point set $\{0\}$ and incrementally add a point of the outer boundary of the current cluster according to some distribution. What we get is a randomly, point by point growing connected cluster which we will call $\mathit{incremental\ aggregate}$. Define 
\begin{align}
	\mathcal{P}_f := \{A\subset \mathbb{Z}^d\ |\ \text{A is finite}\}, 
\end{align}
the set of finite subsets of $\mathbb{Z}^d$. Furthermore we will be interested in distributions on those sets, so for $A\in \mathcal{P}_f$ we define 
\begin{align}
	\mathcal{D}_A:= \{\mu: \mathbb{Z}^d\to [0,1]\ |\ \mu(y) = 0 \text{ for all } y\notin A\ \text{and}\ \sum_{y\in A} \mu(y) = 1 \}, 
\end{align}
the set of distributions on $A$. Now we define an $\mathit{incremental\ aggregate}$ as follows.  

\begin{definition} \label{incrementalaggregate}
	Let $\mu=(\mu_A)_{A\in \mathcal{P}_f}$ be a family of distributions with $\mu_A\in \mathcal{D}_A$ for all $A\in \mathcal{P}_f$. An $\mathit{Incremental\ Aggregate\ (with\ distribution\ \mu)}$ is a stochastic process $(\mathcal{E}_n)_{n\in{\mathbb{N}}}$ which evolves as follows. The process starts with one point $\mathcal{E}_1 = \{0\}$ (define $y_1 :=0$) at the origin of $\mathbb{Z}^d$. Knowing the process $\mathcal{E}_n$ at time $n$, let $y_{n+1}$ be a random point in $\partial \mathcal{E}_n\in \mathcal{P}_f$ with distribution
	\begin{align}
		\mathbb{P}(y_{n+1} = y\ |\ \mathcal{E}_n) := \mu_{\partial \mathcal{E}_n}(y),\quad y\in \mathbb{Z}^d.
	\end{align}
	We then define $\mathcal{E}_{n+1} := \mathcal{E}_n \cup \{y_{n+1}\}$ and the limit cluster as $\E_\infty := \bigcup_{n\in\N} \E_n$. 
\end{definition} 

The above defines a Markov chain whose state space is the set of finite and connected subsets of $\Z^d$. For all incremental aggregates in this paper we will focus on the $2$-dimensional case $\Z^2$ and will always identify $\R^2$ with $\C$ for a more convenient notation. 



\subsection{Notion of Fractal Dimension and Growth Rate}\label{notion}

\begin{figure}
	\centering
	\includegraphics[height=6.5cm]{neuseeland-squares.png}
	\caption{Box-covering of New Zealands outer cost while decreasing the box size $\varepsilon$} \label{neuseeland}
\end{figure}

The notion of fractal dimension is usually used for sets with uncountable cardinality like continuous curves or surfaces. One way of defining a fractal dimension for curves like for example the cost line of New Zealand (see \autoref{neuseeland}) would be to look at the limit relation between the minimum number of boxes (squares) which we need to cover the cost line and the side length of these boxes. The following definition is motivated by \cite{hausdorff} page 160. If for $\varepsilon>0$ $N(\varepsilon)$ is the mininum number of boxes with side length $\varepsilon$ which we need to cover the cost line, then the so called $\mathit{box\text{-}dimension\ d_b}$ is a constant such that $N(\varepsilon)$ grows as fast as $\varepsilon^{-d_b}$ for letting $\varepsilon$ tend to zero, so the box-dimension is defined as
\begin{flalign} \label{boxdimension}
	d_b := - \lim_{\varepsilon\to 0} \frac{\ln(N(\varepsilon))}{\ln(\varepsilon)}. 
\end{flalign} 

This definition makes sense in many contexts, for example is the box-dimension of straight line segments $1$ and of squares $2$ and so on, so in those cases equal to the topological dimension. As with incremental aggregates we are dealing with finite point sets, this approach of defining a fractal dimension for our clusters is not senseful. It is not difficult to show, that the box-dimension of any finite set is $0$. A helpful detail about the situation with finite sets as the ones we are looking at is that each point of the cluster can actually be interpreted and identified with a unique square since the cluster is living on the grid $\Z^2$. We will precise that in chapter \ref{lha}. So instead of decreasing the sizes of the boxes with which we cover our set of interest, we leave the size of the boxes constant and increase the size of our set by adding points and looking at the limit cluster $\E_\infty$. 
Defining the radius of a finite set $A\in \mathcal{P}_f$ (with $|\cdot|$ the euclidean norm) as 
\begin{flalign*}
	\rad(A) := \max_{x\in A} |x|,
\end{flalign*}


we therefore can identify the relation between the geometrical sizes of the boxes and the cluster as follows. For $n\in\N$ define $\varepsilon_n:=\frac{1}{\EE[\rad(\E_n)]}$. Before $\varepsilon_n$ would have expressed the geometrical relation between one box and the whole cluster which we describe now by the fraction $\frac{1}{\EE[\rad(\E_n)]}$ since a box size is now constantly $1$ and the clusters geometrical size is up to a constant given by $\EE[\rad(\E_n)]$. The minimum number of boxes with size $1$ we need to cover $\E_n$ is always $n$, so $N(\varepsilon_n)=n$ for all $n\in\N$ and therefore we can rewrite the definition of the box-dimension (\ref{boxdimension}) by replacing $\varepsilon_n$ with $\frac{1}{\EE[\rad(\E_n)]}$ and define the $\mathit{(discrete)\ fractal\ dimension}$ of $\E_\infty$ as
\begin{flalign} \label{fractaldimension}
	d_f := \liminf_{n\to\infty} \frac{\ln(n)}{\ln(\EE[\rad(\E_n)])}
\end{flalign}
Why we choose the limes inferior here we will see later. Another way of tackling the intuition for a fractal dimension of discrete sets is by considering subsets of the $d$-dimensional ball $B_d(0,m)$ of radius $m\geq 0$, which is motivated by $\cite{fractalwinter}$ Part II page 98 and $\cite{lawler}$ 2.6 page 82. If we take a finite subset $M$ of that ball, we could assign it the dimension $k$ if its cardinality is of order $m^k$. If we apply the same argument to the clusters $\E_n$ and balls with radius $\EE[\rad(\E_n)]$ for all $n\in\N$, we get that the dimension of $\E_\infty$ could be interpreted as a constant $k$ which fulfills that $n=|\E_n|$ grows as fast as $\EE[\rad(\E_n)]^k$ for letting $n$ tend to infinity. We therefore get the same definition as in (\ref{fractaldimension}). This way of defining a fractal dimension for incremental aggregates strongly correlates with the growth rate of the aggregate which shall indicate how the radius of the cluster evolves while increasing the particle number. We can define the growth rate by looking for the smallest exponent $\alpha$ such that there exists a constant $c>0$ with 
\begin{flalign*}
	\EE [\rad(\E_n)] \leq cn^\alpha
\end{flalign*}
for large $n$. Rewriting this we come to the equivalent inequality
\begin{flalign*}
	\frac{\ln(\EE [\rad(\E_n)])}{\ln(n)} - \frac{\ln(c)}{\ln(n)} \leq \alpha
\end{flalign*}
for large $n$ and we could finally define the growth rate $\alpha_f$ of an incremental aggregate as the smallest value satisfying this inequality, so
\begin{flalign} \label{growthrate}
	\alpha_f := \limsup_{n\to\infty} \frac{\ln(\EE [\rad(\E_n)])}{ln(n)}.
\end{flalign}
Since $\rad(\E_n) \leq n$ for all $n\in\N$ a.s., we get 
\begin{flalign*}
	\alpha_f \leq 1. 
\end{flalign*}
 We further chose the limes inferior in (\ref{fractaldimension}) to simply have 
\begin{flalign} \label{fractaldim}
	d_f = \frac{1}{\alpha_f}
\end{flalign}
and therefore
\begin{flalign*}
	d_f \geq 1.
\end{flalign*}
If we are in $\Z^d$ for $d\in\N$ there is a constant $c>0$ such that $cn^{\frac{1}{d}} \leq \rad(\E_n)$ for all $n\in\N$ a.s. and therefore
\begin{flalign*}
	d_f \leq \liminf_{n\to\infty} \frac{\ln(n)}{cn^{\frac{1}{d}}} = d.
\end{flalign*}
So in total we get the trivial boundaries
\begin{flalign*}
	1\leq d_f \leq d
\end{flalign*}
and
\begin{flalign*}
	\frac{1}{d}\leq \alpha_f \leq 1
\end{flalign*}
for any incremental aggregate in $\Z^d$ for $d\in\N$. 




\newpage
\section{External Diffusion Limited Aggregate}

\subsection{Definition}

External DLA is a model of an Incremental Aggregate as defined above using a very natural family of distributions, called the $\mathit{harmonic\ measures}$. 

\begin{definition} $\mathit{(Harmonic\ Measure)}$ Let $A\subset\Z^d$. The hitting probability of $A$ is the function 
	\begin{flalign*}
		H_A: \Z^d \times A \to [0,1],\quad (x,y) \mapsto H_A(x,y):=\PP_x(S_{T_A^+} = y).
	\end{flalign*}
	In literature you can find the same definition where $T_A$ is used instead of $T_A^+$. Since in the following for finite sets $A\in\mathcal{P}_f$ the limit $|x| \to \infty$ of $H_A(x,y)$ is of interest, $T_A^+$ is chosen for convenience. In fact, for a fixed element $x\in\Z^d$ the function $H_A(x,\cdot)$ defines a measure on $A$ with total mass $\PP_x(T_A^+<\infty)$ and it can be adapted to a probability measure by conditioning the random walk to hit $A$ in finite time. Define
	\begin{flalign*}
		\bar H_A: \Z^d \times A \to [0,1],\quad (x,y) \mapsto \bar H_A(x,y):=\PP_x(S_{T_A^+} = y\ |\ T_A^+<\infty), 
	\end{flalign*} 
	so for fixed $x\in\Z^d$ the function $\bar H_A(x,\cdot)$ defines a probability measure on $A$. Indeed this definition is motivated by $\cite{lawler}$ (Chapter $2$, Definition $2.1$) and in the same chapter it is proved, that for finite sets $A\in\mathcal{P}_f$ the limit
	\begin{flalign*}
		\lim_{|x|\to\infty} \bar H_A(x,y) =: h_A(y) 
	\end{flalign*}
	exists for each $y\in A$. The function $h_A: A\to [0,1]$ is called the $\mathit{harmonic\ measure\ of\ A}$. For an element $y\in A$, $h_A$ can be interpreted as the probability that a random walk starting at $\glqq \text{infinity}\grqq$ hits $A$ the first time at $y$. If we look at the $2$-dimensional case, by Lemma $\autoref{recurr}$, we have that a random walk is recurrent on $\Z^2$ and therefore we get $\bar H_A = H_A$ since $\PP_x(T_A^+<\infty) = 1$ by recurrence. 
\end{definition}

\begin{definition} $\mathit{(External\ Diffusion\ Limited\ Aggregate)}$ $\mathit{External\ Diffusion\ Limited}$ $\mathit{Aggregate}$, short $\mathit{External\ DLA}$, is an incremental aggregate with the family of harmonic measures $(h_A)_{A\in\mathcal{P}_f}$ as distribution. 
\end{definition}


\subsection{Fractal Dimension and Growth Rate of External DLA}

Looking at computer simulations it seems that the clusters of External DLA are formed relatively sparse and they appear to have a noninteger fractal dimension. In $\cite{magnetic}$ DLA is observed in a magnetic aggregation context, and empirically they find a fractal dimension of around $1.8$. Other simulations seem to suggest a value little less than 1.7 for $d_f$ in two dimensions ($\cite{lawler}$ page 83). There is also a theory that predicts 
\begin{flalign*}
	d_f = \frac{d^2 + 1}{d+1}
\end{flalign*}
in $\Z^d$ which seems to agree fairly well with simulations ($\cite{lawler}$ page 83). There are only few rigorously prooved results and we will see one of them in the following. First we will proof some lemmas. For that define two random variables
	\begin{flalign*}
		r: \N \to [0,\infty),\quad n\mapsto \rad(\E_n)
	\end{flalign*}
	and
	\begin{flalign*}
		T: [0,\infty) \to \N,\quad s\mapsto \min\{j\in\N\ |\ r(j)\geq s\}.
	\end{flalign*}
Then it is easy to show that for all $n\in\N$, $s\in [0,\infty)$ and $(\omega)\in\Omega$
	\begin{enumerate} \label{props}
		\item $r(\omega)$ and $T(\omega)$ grow monotonously 
		\item $T(\omega)(r(\omega)(n)) \leq n$
		\item $r(\omega)(T(\omega)(s)) \geq s$
	\end{enumerate}
hold. 

\begin{lemma} \label{rtinfty}
	For both random functions $r$ and $T$ we have that
	\begin{flalign*}
		r(\omega)(n) \to\infty \text{ for } n\to\infty
	\end{flalign*}
	and
	\begin{flalign*}
		T(\omega)(s) \to\infty \text{ for } s\to\infty
	\end{flalign*}
	for all $\omega\in\Omega$.
\end{lemma}
\begin{proof}
	Since we are moving on the grid $\Z^d$ we have that for a ball $B_d(0,n)$ with radius $n\geq0$ we have a finite number $N:=|B_d(0,n)\cap \Z^d|$ and for any $\omega\in\Omega$ get 
	\begin{flalign*}
		r(\omega)(2N)\geq n. 
	\end{flalign*}
	Therefore for all $\omega\in\Omega$ and any $n\in\N$ we can find a $N\in\N$ such that $r(\omega)(N)\geq n$, and since $r(\omega)$ grows monotonously we get $r(\omega)(n) \to\infty \text{ for } n\to\infty$ for all $\omega\in\Omega$. Very similarly we can argument for $T$. 
\end{proof}

\begin{lemma} \label{randt}
	Let $a>0$ and $h:[0,\infty) \to [0,\infty)$ a bijective, multiplicative and monotonously growing function. Then the following are equivalent:
	\begin{enumerate}
		\item $\exists c>0: \PP(r(n) \leq ch(n) \text{ for large $n$}) = 1$ 
		\item $\exists c>0: \PP(T(as)\geq ch^{-1}(s) \text{ for large $s$})=1$
	\end{enumerate}
\end{lemma}

\begin{proof} 
	$\Rightarrow$: 
	For $c>0$ define
	\begin{flalign*}
		A_c:=\{r(n) \leq ch(n) \text{ for large $n$}\}
	\end{flalign*}
	and
	\begin{flalign*}
		B_c := \{T(as)\geq ch^{-1}(s) \text{ for large $s$}\}.
	\end{flalign*}
	Choose $c>0$ such that $\PP(A_c)=1$. Take $\omega\in A_c$ and choose $N\in\N$ such that $r(\omega)(n)\leq ch(n)$ for all $n>N$. Hence there exists $\tilde c>0$ such that. $\tilde ch^{-1}(r(\omega)(n))\leq n$ for all $n>N$. By Lemma \ref{rtinfty} we can choose $M\in\N$ big enough such that $T(\omega)(aM) > N$, hence $T(\omega)(as)\geq T(\omega)(aM) > N$ for all $s>M$ since $T(\omega)$ grows monotonously. Hence we can write $\tilde c h^{-1}(r(T(\omega)(as))) \leq T(\omega)(as)$ for all $s>M$ and since $r(T(\omega)(as))\leq as$ we finally get $\tilde c h^{-1}(a)h^{-1}(s)=\tilde ch^{-1}(as) \leq T(\omega)(as)$ for all $s>M$, hence $\omega \in B_{\tilde c h^{-1}(a)}}$, where we used the multiplicativity of $h$ and that $h^{-1}$ falls monotonously. We therefore get $A_c\subset B_{\tilde c h^{-1}(a)}$, hence $\PP(B_{\tilde c h^{-1}(a)}) = 1$.\\
	$\Leftarrow$: 
	The proof for this direction works analogously.
\end{proof}

\begin{lemma} \label{geometric}
	Let $n\in\N$ and $T_1,\dots,T_n$ be independent geometric random variables with parameter $p<\frac{1}{2}$. Let $Y=T_1 + \dots  + T_n$, then for every $a \in [2p,1)$ we have
	\begin{flalign*}
		\PP(Y\leq\frac{an}{p}) \leq (ae^2)^n. 
	\end{flalign*}
\end{lemma}

\begin{proof}
	The moment generating function of $Y$ is
	\begin{flalign*}
		\EE[e^{tY}] = (pe^t)^n(1-e^t(1-p))^{-n} = p^n(e^{-t} - (1-p))^{-n}.
	\end{flalign*}
	By Chebyshev for any random variable $X$ we know the inequality
	\begin{flalign*}
		\PP(X\geq x) \leq \inf_{t>0} \frac{\EE[e^{tX}]}{e^{tx}}
	\end{flalign*}
	and can therefore follow that for any $t>0$ 
	\begin{flalign*}
		\PP(Y\leq \frac{an}{p}) = \PP(-Y\geq -\frac{an}{p}) \leq \exp(\frac{ant}{p})\EE[e^{-tY}] = \exp(\frac{ant}{p})p^n(e^t- (1-p))^{-n}. 
	\end{flalign*} 
	Choose $t=\ln(\frac{a(1-p)}{a-p})$ (note that $t>0$), then
	\begin{flalign*}
		\PP(Y\leq \frac{an}{p}) &\leq (\frac{a(1-p)}{a-p})^{\frac{an}{p}}p^n(1-p)^{-n}(\frac{p}{a-p})^{-n} \\
		&= (\frac{a}{a-p})^{\frac{an}{p}} (1-p)^{\frac{an}{p}} (1-p)^{-n}(\frac{1}{a-p})^{-n} \\
		&\leq (1 + \frac{p}{a-p})^{\frac{an}{p}}(1-p)^{n}(1-p)^{-n}(a-p)^{n} \\
		&\leq (1 + \frac{p}{a-p})^{\frac{2(a-p)n}{p}}a^n \\
		&\leq (ae^2)^n.
	\end{flalign*}
\end{proof}

\begin{definition}
	For $x\in\Z^2$ and $n\in\N$ define 
	\begin{flalign*}
		{\mP}^x_n := \{ A\in{\mP}_f\ |\ x\in A, n=\max_{y\in A} |x-y|\text{ and } A \text{ is connected}\}.
	\end{flalign*}
\end{definition}

The following theorem gives a boundary on the harmonic measure which will be necessary to proof the result for the growth rate of External DLA. The theorem is proved by $\cite{lawler}$ in Theorem $2.5.2$ and we state it slightly different here to generalize it easier afterwards. 

\begin{theorem} \label{keytheorem}
	There exists a constant $c>0$ such that for all $n\in\N$
	\begin{flalign*}
		h_A(0) \leq cn^{-\frac{1}{2}} \quad \text{ for all } A\in\mP^0_n.
	\end{flalign*}
\end{theorem}
\begin{proof}
	If $A\in\mP^0_n$ then $0\in A$ and $n=\rad(A)$ and for that case the proof is presented for all dimensions $d\in \N$ in $\cite{lawler}$ Theorem $2.5.2$. The proof is very technical and requires various results of other theorems and lemmas which are also to find in $\cite{lawler}$. 
\end{proof}

\begin{proposition}
	Theorem \ref{keytheorem} can be generalized to the following. Let $x\in\Z^2$. Then there exists $c>0$ such that for all $n\in\N$
	\begin{flalign*}
		h_A(x) \leq cn^{-\frac{1}{2}} \quad \text{ for all } A\in\mP^x_n.
	\end{flalign*}
\end{proposition}

\begin{proof}
	
\end{proof}


\begin{theorem}
	For the growth rate of External DLA in $\Z^2$ according to the definition in $\ref{notion}\ (\ref{growthrate})$ we have
	\begin{flalign}
		\alpha_f \leq \frac{2}{3}. 
	\end{flalign}	
\end{theorem}
\begin{proof}
	For $c>0$ define 
	\begin{flalign*}
		A_c := \{\rad(\E_n) \leq cn^{\frac{3}{2}} \text{ for large }n\}
	\end{flalign*}
	and
	\begin{flalign*}
		D_c := \{T(2n) \geq c n^{\frac{2}{3}} \text{ for large }n \}.
	\end{flalign*}
	If we have a $c>0$ such that $\PP(A_c) = 1$, then we have
	\begin{flalign*}
		\alpha_f &= \limsup_{n\to\infty} \frac{\ln(\EE[\rad(\E_n)])}{\ln(n)}\\ &= \limsup_{n\to\infty} \frac{\ln(\int_\Omega \rad(\E_n) d\PP)}{\ln(n)} \\
		&= \limsup_{n\to\infty} \frac{\ln(\int_{A_c} \rad(\E_n) d\PP)}{\ln(n)} \\
		&\overset{(+)}{\leq} \limsup_{n\to\infty} \frac{\ln(\int_{A_c} cn^{\frac{3}{2}} d\PP)}{\ln(n)} \\ 
		&= \limsup_{n\to\infty} \frac{\ln(cn^{\frac{3}{2}})}{\ln(n)} \\
		&= \frac{3}{2} \limsup_{n\to\infty} \frac{\ln(c^{\frac{2}{3}}) + \ln(n)}{\ln(n)} = \frac{3}{2} \cdot 1 = \frac{3}{2}
	\end{flalign*}
	If we define
	\begin{flalign*}
		h: [0,\infty) \to [0,\infty), x\mapsto x^{\frac{2}{3}},  
	\end{flalign*}
	and $a=2$ by Lemma \ref{randt} we can show the above if there exists a constant $c>0$ such that 
	\begin{flalign} \label{lambda}
		\PP(D_c) = 1. 
	\end{flalign}
	Note that $h$ is bijective, multiplicative and monotonously growing. Lets first argument why the inequality at $(+)$ indeed holds. We define ...
	
	
	So now we will try to prove (\ref{lambda}). For $n\in\N$ write $\E_n = \{y_1,\dots,y_n\}$ according to the definition in $\ref{incrementalaggregate}$, where $y_j$ is the $j$-th point added to the cluster. Let $\beta > 0$ which will be determined lateron. For $n\in\N$ let $\tilde m_n := \beta n^{\frac{3}{2}}$ and define 
	\begin{flalign*}
		V_n := \{T(2n) < \tilde m_n\}. 
	\end{flalign*}
	Further define the set of realised random walk paths of length $n$ with starting point in $\partial B_n$ 
	\begin{flalign*}
	Z_n := \{[z] := \{z_1, \dots, z_n\}\subset \Z^2\ |\ z_1\in \partial B_n \text{ and } z_i\in N(z_{i-1}) \text{ for all } i\in\{2,\dots,n\}\}, 
	\end{flalign*}
	for $[z]\in Z_n$ define events 
	\begin{flalign*}
		W_n([z]) := \{\omega\in\Omega\ |\ \exists j_1< \dots < j_n \leq \tilde m_n \ \text{ such that } y_{j_i}(\omega)  = z_i \text{ for all } i\in\{1,\dots,n\} \}
	\end{flalign*}
	and the union of these events 
	\begin{flalign*}
		W_n := \bigcup_{[z] \in Z_n} W_n([z]). 
	\end{flalign*}
	Since the random walks are moving on the grid $\Z^2$, by $z_1\in\partial B_n$ we mean that $z_1$ is $\glqq$near$\grqq$ to $\partial B_n\subset\R^2$, precisely we could mean $z_1\in (B_n\setminus B_{n-1}) \cap \Z^2$. Anyway this detail will not create problems of relevant order. We will quickly proof that 
	\begin{flalign*}
		V_n \subset W_n \text{ for all } n\in\N.
	\end{flalign*}
	Let $n\in\N$ and $\omega \in V_n$, then $T(2n)(\omega) < \tilde m_n$. For $m_n:=\max\{j\in\N\ |\ j\leq \tilde m_n\}$ we therefore have $\rad(\E_{m_n}(\omega)) \geq 2n$ (Note that $T(2n)(\omega)\in\N$). Therefore since $\E_{m_n}(\omega)$ is connected there must exist indices $j_1 <\dots < j_n$ s.t. $y_{j_1}(\omega)\in \partial B_n$ and $[z_0] := [y_{j_1}(\omega),\dots, y_{j_n}(\omega)]\subset \E_{m_n}(\omega)$, since $\max_{x\in [z_0]} |x| \leq 2n$. Therefore $j_n \leq m_n\leq\tilde m_n$ and therefore $\omega \in W_n([z_0]) \subset W_n$ which proofs the inclusion. For a sequence of events $(A_n)_{n\in\N}$ remind this definition
	\begin{flalign*}
		\limsup_{n\to\infty} A_n := \bigcap_{n\in\N} \bigcup_{i\geq n} A_i = \{\omega\in\Omega\ |\ \omega \in A_n \text{ for infinitely many } n\in\N\}.
	\end{flalign*}
	We will want to use the Lemma of Borel-Cantelli on the sequence of events $(W_n)_{n\in\N}$. If we can show that 
	\begin{flalign} \label{borelcantelli}
		\sum_{n\in\N} \PP(W_n) < \infty,
	\end{flalign}
	then with $V_n\subset W_n$ for all $n\in\N$ and Borel-Cantelli we get
	\begin{flalign*}
		\PP(\limsup_{n\to\infty} V_n) \leq \PP(\limsup_{n\to\infty} W_n) = 0. 
	\end{flalign*}
	Since 
	\begin{flalign*}
		(\limsup_{n\to\infty} V_n)^C = \{\omega\in\Omega\ |\ \exists N\in\N \text{ s.t. } \omega\in V_n^C \text{ for all }n>N\}= D_\beta
	\end{flalign*}
	we then can follow that $\PP(D_\beta) = 1$ and have finished the proof. So we want to show $(\ref{borelcantelli})$. For $n\in \N$, $[z]\in Z_n$ and $i\in \{1,\dots,n\}$ define random variables
	\begin{flalign*}
		\tau_i:\Omega \to \N^\infty, \tau(\omega) = j :\Leftrightarrow y_j(\omega) = z_i, 
	\end{flalign*}
	so $\tau_i$ is the index $j$ such that the $j$-th added point to the cluster is equal to $z_i$. This definition of $\tau_i$ is well-defined since for every $\omega\in\Omega$ there can only exist one and certainly exists at least one $j\in\N^\infty$ such that $y_j(\omega) = z_i$ as the final cluster $\E_\infty$ either contains $z_i$ or not. Note that it is possible that $z_i$ never gets added to the cluster for example when the cluster forms such that $z_i$ is inside a hole of the cluster, so $\PP(\tau_i = \infty) > 0$. Further we define waiting times
	\begin{flalign*}
		\sigma_i: \Omega \to \N^\infty, \omega\to \begin{cases}
			\tau_{i+1}(\omega) - \tau_i(\omega), &\text{ if } \tau_{i+1}(\omega) < \infty \text{ and } \tau_i(\omega)<\infty, \\
			\infty, &\text{ else},	
		\end{cases}
	\end{flalign*}
	so $\sigma_i$ is the waiting time between adding $z_i$ and $z_{i+1}$ to the cluster. Consider the event $W_n([z])$ and take $i\in\{1,\dots,n\}$. Note that in this case $\tau_i$ is finite for all $i\in\{1,\dots,n\}$. Since $\E_{j_i}$ is connected and contains $0$, by Theorem $\ref{keytheorem}$ we get that the distribution of $\sigma_i$ is bounded by the one of a geometric random variable with parameter
	\begin{flalign*}
		p_n := c_1 n^{-\frac{1}{2}}
	\end{flalign*}
	for some constant $c_1>0$. Now choose $\beta$ such that $4e^2\beta c_1 < 1$ und choose $N\in\N$ such that $\beta c_1 \geq 2p_n$ for all $n>N$. If we then define $a:=\beta c_1$ and $Y:=\sum_{i=1}^{n-1} \sigma_i = \tau_{n} - \tau_1$ we can use Lemma \ref{geometric} and get
	\begin{flalign*}
		PROBLEM
		\PP(\tau_{n} \leq \tilde m_n) \leq \PP(\tau_{n} - \tau_1 \leq \tilde m_n) = \PP(Y \leq \frac{an}{p_n}) \leq (e^2a)^{n-1} \quad \text{ for all } n>N
	\end{flalign*}
	and since $W_n([z]) \subset \{\tau_{n} \leq \tilde m_n\}$ we get
	\begin{flalign*}
		\PP(W_n([z])) \leq (e^2a)^{n} \quad \text{ for all } n>N.
	\end{flalign*}
	Counting the elements in $Z_n$ we have less or equal $c_2n$ points in $\partial B_n$ for some constant $c_2>0$ as starting points and $4^{n-1}$ possibilities for the next $n-1$ steps of a random walk of length $n$. So $|Z_n| \leq c_2n4^{n-1}$ and therefore
	\begin{flalign*}
		\sum_{[z]\in Z_n} \PP(W_n([z])) \leq c_2n4^{n-1} (e^2a)^{n-1} = c_2 n(4e^2a)^{n-1} \quad \text{ for all } n>N
	\end{flalign*} 
	and finally
	\begin{flalign*}
		\sum_{n>N} \PP(W_n) \leq \sum_{n>N} c_2 n(4e^2\beta c_1)^{n-1} =: r_\beta. 
	\end{flalign*}
	Since $\beta$ was chosen such that $4e^2\beta c_1<1$, $r_\beta$ is finite and therefore in total get
	\begin{flalign*}
		\sum_{n\in\N} \PP(W_n) < \infty, 
	\end{flalign*}
	which completes the proof. 
\end{proof}

\begin{remark}
	With this theorem and \ref{fractaldim} we conclude that the fractal dimension of External DLA is bigger or equal than $\frac{3}{2}$. In the next chapter we will look at another incremental aggregate which tries to approximize External DLA and compare simulations of both to get a hint on which growth rate of the two is bigger than the other one.
\end{remark}















\newpage
\section{Line Hitting Aggregate} \label{lha}

\subsection{Motivation}

In the following we will look at a process which is the approach of a simple approximation of external DLA on $\mathbb{Z}^2$. The idea is to let particles move on straight lines coming from infinity and add them to the cluster where they hit it. Obviously in most cases particles cannot move completely straight on $\mathbb{Z}^2$. Therefore we will consider points in $\mathbb{Z}^2$ as the centers of unit squares and let the particles move on straight lines in the full plane $\mathbb{R}^2$. We consider a line hitting a point in $\mathbb{Z}^2$ if and only if it intersects with its unit square as defined in the following. 

\begin{definition} \label{squares}
	Define 
	\begin{align}
		\C_{sq} := \{[k - \frac{1}{2}, k + \frac{1}{2}] + [l- \frac{1}{2}, l + \frac{1}{2}]i \subset \C\ |\ k,l \in \mathbb{Z}\}, 
	\end{align} 
	note that $\C = \bigcup_{s\in \C_{sq}} s$. The canonical function
	\begin{align}
	sq: \mathbb{Z}^2 \to \C_{sq},\quad (k,l)\to [k - \frac{1}{2}, k + \frac{1}{2}] + [l- \frac{1}{2}, l + \frac{1}{2}]i
	\end{align}
	is bijective and intuitively identifies points in $\mathbb{Z}^2$ with squares in $\C$ which is $p$ is the center of the square $sq(p)$ for all $p\in \mathbb{Z}^2$. In the following when using a point $p\in \mathbb{Z}^2$ it will reference the point in $\mathbb{Z}^2$ or the corresponding square in $\C$ respecting the context. This bijection also naturally defines a graph structure on $\C_{sq}$, which is two squares $s_1, s_2\in \C_{sq}$ form an edge if and only if $sq^{-1}(s_1)$ and $sq^{-1}(s_2)$ form an edge in $\mathbb{Z}^2$. 
	\noindent For the following we say a line $g$ $hits$ a point $p\in \mathbb{Z}^2$ if and only if $g\cap sq(p) \neq \emptyset$ (see in $\autoref{linesquares}$). 
	
\end{definition}

\begin{figure}
	\centering
	\includegraphics[height=10cm]{line-hit-squares.png}
	\caption{$g$ hits squares around points} \label{linesquares}
	\includegraphics[height=10cm]{line-hit-A.png}
	\caption{$g$ hits squares in $A$} \label{linesquaresA}
\end{figure}

\begin{definition}
	Let $g=g_{\alpha,p}\in \G$ and $A\in \mathcal{P}_f$. We define 
	
	\begin{align*}
		g\cap A := \{ p\in A\ |\ g \text{ hits } p\}
	\end{align*}
	
	which is the subset of all points in $A$ which are hit by $g$ (see $\autoref{linesquaresA}$). For the following we suppose $g\cap A \neq \emptyset$. We will define a total ordered relation $\triangleleft$ on $g\cap A$ which shall be defined equivalently for all $g\in \G$ and $A\in\mathcal{P}_f$ with $g\cap A \neq\emptyset$. We choose two points $x,y\in g\cap A$ and split the definition of the relation $\triangleleft$ into four cases, depending on whether the line $g$ goes from left-bottom to right-top, left-top to right-bottom, parallel to the $x$-axis and parallel to the $y$-axis. Denote the real part of $x$ with $Re(x)$ and the imaginary one with $Im(x)$. \\
	\\
	$\mathit{Case}\ 1:\quad g\ \text{ is parallel to the x-axis}\quad (\Leftrightarrow\quad \alpha = \frac{\pi}{2})$
	\begin{align*}
	x \triangleleft y \quad :\Leftrightarrow \quad Re(x) < Re(y)
	\end{align*}\\
	$\mathit{Case}\ 2:\quad g\ \text{ is parallel to the y-axis}\quad (\Leftrightarrow\quad \alpha = 0)$
	\begin{align*}
	x \triangleleft y \quad :\Leftrightarrow \quad Im(x) < Im(y)
	\end{align*}\\
	$\mathit{Case}\ 3:\quad g\ \text{ is going from left-bottom to right-top}\quad (\Leftrightarrow\quad \alpha\in (\frac{\pi}{2},\pi))$
	\begin{align*}
	x \triangleleft y \quad :\Leftrightarrow \quad
		\begin{cases}
			Re(x) < Re(y), & \text{ if } Re(x) \neq Re(y), \\
			Im(x) < Im(y), & \text{ if } Re(x) = Re(y).
		\end{cases}
	\end{align*}\\
	$\mathit{Case}\ 4:\quad g\ \text{ is going from left-top to right-bottom}\quad (\Leftrightarrow\quad \alpha\in (0,\frac{\pi}{2}))$
	\begin{align*}
	x \triangleleft y \quad :\Leftrightarrow \quad
	\begin{cases}
	Re(x) < Re(y), & \text{ if } Re(x) \neq Re(y), \\
	Im(x) > Im(y), & \text{ if } Re(x) = Re(y).
	\end{cases}
	\end{align*}\\
	It is easy to see that this relation on $g\cap A$ is well-defined. In the following we will prove that this relation is strictly and totally ordered. 
\end{definition}

\begin{lemma}
For a line $g=g_{\alpha,p}\in\G$ and $A\in \mathcal{P}_f$ with $g\cap A\neq \emptyset$ the relation $\triangleleft$ on $g\cap A$ is totally ordered. 
	\begin{proof}
		We will only proove the case where $g$ is going from left-bottom to right-top, which is $\mathit{Case}\ 3$ of the definition. In this case we have $\alpha\in (\frac{\pi}{2},\pi)$. Note, that the proof for $\mathit{Case\ }4$ will work very similar and in the case of $g$ being parallel to one of the axes ($\mathit{Case\ }1$ or $2$), all properties for a totally ordered relation follow directly from the totally ordered relation $<$ on $\mathbb{R}$. So let $\alpha\in (\frac{\pi}{2},\pi)$. \\
		\\
		$\mathit{Antisymmetry:}$ For antisymmetry let $x \triangleleft y$ and $y \triangleleft x$. Suppose $Re(x)\neq Re(y)$, then $Re(x) < Re(y)$ and $Re(y) < Re(x)$, a contradiction because of the total order $<$ in $\mathbb{R}$. So $Re(x) = Re(y)$. But then we have $Im(x) < Im(y)$ and $Im(y) < Im(x)$ and therefore also $Im(x) = Im(y)$, hence $x=y$. \\
		\\
		$\mathit{Transitivity:}$ For transitivity let $x \triangleleft y$ and $y \triangleleft z$. We find four cases. In case $Re(x) \neq Re(y)$ and $Re(y) \neq Re(z)$ we get $Re(x) < Re(z)$ by transitivity of $<$, hence $x \triangleleft z$. In case $Re(x)\neq Re(y)$ and $Re(y) = Re(z)$ we get $Re(x) < Re(y) = Re(z)$, therefore $x \triangleleft z$. In case $Re(x) = Re(y)$ and $Re(y) \neq Re(z)$ we get $Re(x) = Re(y) < Re(z)$, similar as the last case. In the last case $Re(x) = Re(y) = Re(z)$ we get $Im(x) < Im(y)$ and $Im(y) < Im(z)$ and again by transitivity of $<$ we get $Im(x) < Im(z)$, hence $x \triangleleft z$ again. \\
		\\
		$\mathit{Connexity:}$ Connexity is given since for any two points $x,y\in g\cap A$ we have either $Re(x) \neq Re(y)$ or $Re(x) = Re(y)$ and therefore either $x\triangleleft y$ or $y\triangleleft x$.
	 
	\end{proof}
\end{lemma}

\begin{remark}
	The relation $\triangleleft$ on $g\cap A$ basically orders the hitting points of $g$ with $A$ from left to right (or bottom to top in case of a line parallel to the $y$-axis). This order allows us to identify the outermost hitting points which are the minimum and maximum of $g\cap A$ with respect to $\triangleleft$. To clarify, we define $\min (g\cap A) := x_0$ if and only if $x_0 \triangleleft x$ for all $x\in g\cap A,x\neq x_0$, analogously $\max(g\cap A)$. This means when moving on $g$ facing $A$ coming from infinity this order allows to know where in $A$ the line $g$ hits first when $\glqq$entering$\grqq$ $A$ and where it hits last when $\glqq$leaving$\grqq$ $A$. What we want to do next is to choose a line randomly out of all lines which hit the current cluster. This is isn`t an obvious task and we will have to develop a most fair underlying distribution on lines in the next section where we will touch basic aspects of Integral Geometry. 
\end{remark}






\subsection{Integral Geometry}

In the next section we want to define an approximation for External DLA. This approximation will be an incremental aggregate for which distribution definition we need some concepts and results from Integral Geometry which we will discuss and develop in this section. In the process we want to define we will want to choose a random line out of all lines which intersect with the current cluster of the aggregate. This random choosing is not obvious since most of the time the cluster will be strongly non symmetric and it is even less obvious how to actually get a realisation of a random line when simulating with Python. In our case we are looking for a parametrisation of lines in the plane and a reasonable way of choosing parameters randomly. \\

We will introduce a possible solution for this problem first through the abstract and general concepts of integral geometry and later through a simple parametrisation for the case of lines in the plane which goes hand in hand with the general result.

\subsubsection{General results}

In the general context we are in $\R^d$ for $d\in \N$ and consider $q$-dimensional affine subspaces where $q\in \{0,\dots,d\}$, short $q$-flats in $\R^d$. The set of $q$-flats in $\R^d$ is denoted by $A(d,q)$. Later we will be interested in choosing random lines in the real plane (i.e. $1$-flats in $\R^2$). In order to get a probability measure on some set of $q$-flats, we first need a measure and a $\sigma$-algebra on $A(d,q)$ in total. 

\begin{definition}
	For $B\in \mathcal{B}^d$ define 
	\begin{flalign*}
		[B]_{d,q} := \{F\in A(d,q)\ |\ F\cap B \neq\emptyset\}.
	\end{flalign*}
	If the context is clear, we will only write $[B]$ instead of $[B]_{d,q}$. 
\end{definition}

\begin{definition}
	The $\sigma$-algebra $\mathcal{A}(d,q)$ on $A(d,q)$ is defined by
	\begin{flalign*}
		\mathcal{A}(d,q) := \sigma(\{ [K]\ |\ K\in \K^d\}).
	\end{flalign*} 
\end{definition}

\begin{theorem} \label{uniqmeas}
	On $A(d,q)$ there exists a unique $G_d$-invariant Radon measure $\mu_q$ such that
	\begin{flalign}
		\mu_q(A_{B_d(1,0)}) = \kappa_{d-q}, 
	\end{flalign}
	where $\kappa_n := \lambda_n(B_n(1,0))$ is the $n$-dimensional Lebesque meausure of the $n$-dimensional unit ball for $n\in \N$, and $\kappa_0:=1$.
\end{theorem}
\begin{proof}
	\cite{stoch1} Theorem 4.26 \\ \\ MAKE PROOF CLEARER
\end{proof}

\subsubsection{Construction in the plane: Isotropic lines}

For our special case we choose $d=2$ and $q=1$, thus lines in the plane. We denote this set of lines by $\G$. The following construction in this chapter is completely motivated by \cite{sackmann} 2.1.1. Firstly we propose a parametrisation of lines which works as follows. Every line can be uniquely determined by an angle $\alpha\in [0,\pi)$ and a real number $p\in \R$. Let $\langle\cdot,\cdot\rangle$ be the standard scalar product on $\R^2$, respectively used for values in $\C$ as we identify $\R^2$ with $\C$ as $\R$-vectorspaces. Let $e_\alpha := e^{\alpha i} = \cos(\alpha) + \sin(\alpha)i$ and $s_\alpha : = -\sin(\alpha) + \cos(\alpha)i$ be the unit vectors $1$ and $i$ rotated by $\alpha$ counterclockwise. Lets consider the representation $x = \langle x,e_\alpha\rangle e_\alpha + \langle x,s_\alpha\rangle s_\alpha$ for $x\in \C$. Since $e_\alpha$ and $s_\alpha$ form a base of $\C$ as a $\R$-vectorspace, the parameters $\langle x,e_\alpha\rangle$ and $\langle x, s_\alpha\rangle$ are unique for each $x$. It thus is easy to realize that $g_{\alpha,p} := \{x\in \C\ |\ \langle x,e_\alpha\rangle  = p\}$ defines a line (compare with $\autoref{lineparam}$) and that every line has a unique pair of $\alpha$ and $p$ for such a representation. In words, $g_{\alpha,p}$ contains all points which have length $p$ in direction of $e_\alpha$. With $\Phi := [0,\pi) \times \R$ this naturally defines a bijection
\begin{flalign*}
	\chi: \Phi \to \G, \quad (\alpha,p) \mapsto g_{\alpha,p}. 
\end{flalign*}
\\
\begin{figure}
	\centering
	\includegraphics[height=10cm]{line-param.png}
	\caption{Line parameters $\alpha$ and $p$} \label{lineparam}
\end{figure}
\\


We take the subspace Borel-$\sigma$-algebra $\mathcal{B}_\Phi:= \mathcal{B}^2 \cap \Phi$ on $\Phi$ and define the $\sigma$-algebra $\GG$ on $\G$ by $\GG := \chi(\mathcal{B}_\Phi)$. This works well since $\chi$ is a bijection. We want to show in the following that this way of defining a $\sigma$-algebra on $\G$ makes sense as it is indeed equivalent to the general context as defined above. To do that it is convenient to use a special generator set for the Borel-$\sigma$-algebra $\mathcal{B}_\Phi$. 

\begin{lemma} \label{generators}
	Define 
	\begin{flalign*}
		\mathcal{R}_+ := \{[\alpha,\beta]\times (0,b]\ |\ 0\leq \alpha<\beta<\pi, b \geq 0\}, 
	\end{flalign*}
	\begin{flalign*}
		\mathcal{R}_- := \{[\alpha,\beta]\times [b,0)\ |\ 0\leq \alpha<\beta<\pi, b\leq 0\},
	\end{flalign*}
	\begin{flalign*}
		\mathcal{R}_0 := \{[\alpha,\beta]\times \{0\}\ |\ 0\leq \alpha<\beta<\pi\},
	\end{flalign*}
	and
	\begin{flalign*}
		\mathcal{R} := \mathcal{R}_+ \cup \mathcal{R}_- \cup \mathcal{R}_0.
	\end{flalign*}
	Then $\sigma(\mathcal{R}) = \mathcal{B}_\Phi$.
\end{lemma}
\begin{proof}
	We show that $\sigma(\mathcal{R})$ contains all rectangles in $\Phi$ of the form $[\alpha,\beta]\times (a,b]$ with $a,b>0$, $[\alpha,\beta]\times [a,b)$ with $a,b<0$ and $[\alpha,\beta]\times [a,b]$ with $0\in [a,b]$. First let $a,b>0$ and $R=[\alpha,\beta]\times (a,b]$, then 
	\begin{flalign*}
		R = ([\alpha,\beta] \times (0,b]) \setminus ([\alpha,\beta] \times (0,a])
	\end{flalign*}
	and therefore $R\in \sigma(\mathcal{R}_+)\subset\sigma(\mathcal{R})$. Similarly it works if $a,b<0$. If $0\in[a,b]$ then we can write $R = [\alpha,\beta]\times [a,b]$ with three components $R = [\alpha,\beta] \times [a,0) \cup [\alpha,\beta] \times \{0\} \cup [\alpha,\beta] \times (0,b]$ which lie in $\mathcal{R}_-, \mathcal{R}_0$ and $\mathcal{R}_+$ respectively. Therefore $R\in \sigma(\mathcal{R})$ aswell. By measure theory the above described rectangles form a generator set of $\mathcal{B}_\Phi$, which completes the proof. 
\end{proof}

\begin{lemma}
	We have $\mathcal{A}(2,1) = \GG$. 
\end{lemma}
\begin{proof}
	We will consider generators of these $\sigma$-algebras. By Lemma $\autoref{generators}$ we know that $\sigma(\mathcal{R}) = \mathcal{B}_\Phi$ and since $\chi$ is a bijection, we have $\chi(\sigma (\mathcal{R})) = \sigma (\chi(\mathcal{R}))$ and finally $\GG = \sigma(\chi(\mathcal{R}))$. For $\tilde A := \{[K]\ |\ K\in \K^2\}$ we have by definition $\mathcal{A}(2,1) = \sigma(\tilde A)$. \\
	%
	\\ \indent $\subset$: Let $K\in\K^2$. We will show that $\chi^{-1}([K])$ is a closed set in $\Phi$. If that is the case we have $\chi^{-1}([K])\in\mathcal{B}_\Phi$, therefore $[K] \in\chi(\mathcal{B}_\Phi) = \GG$ and finally $\mathcal{A}(2,1) = \sigma(\tilde A)\subset\GG$. To show that $\chi^{-1}([K])$ is closed let $(\alpha_0,p_0)\in \Phi\setminus \chi^{-1}([K])$. Then $\chi(\alpha_0,p_0) \notin [K]$ and therefore $\chi(\alpha_0,p_0) \cap K= \emptyset$. Since $K$ is closed we can find small values $\tilde\alpha,\tilde p > 0$ such that $\chi(\alpha,p) \cap K = \emptyset$ for all $(\alpha,p)\in [\alpha_0, \alpha_0 + \tilde\alpha] \times [p_0, p_0 + \tilde p] =: R$. Hence we have $ R\subset \Phi \setminus \chi^{-1}([K])$, so $\Phi \setminus \chi^{-1}([K])$ is open. Hence $\chi^{-1}([K])$ is closed. \\
	%
	\\ \indent $\supset$: For this inclusion we will show that $\chi(R)\in\mathcal{A}(2,1)$ for all $R\in\mathcal{R}$. First let $R = [\alpha,\beta]\times(0,b]\in\mathcal{R}_+$ for some $b>0$. Define 
	\begin{flalign*}
		S:= \{pe_\gamma \in \C\ |\ (\gamma,p)\in R\}
	\end{flalign*}
	Furthermore for $n\in \N$ define 
	\begin{flalign*}
		A_n := \{tns_\beta\ |\ t\in [0,1]\} \text{ and } B_n := \{-tns_\alpha\ |\ t\in [0,1]\},
	\end{flalign*}
	the segments from $0$ to $ns_\beta$ and $0$ to $-ns_\alpha$ ($\autoref{circleS}$). We will show now that 
	\begin{flalign*}
		\chi(R) = [\bar S] \setminus (\bigcup_{n\in\N} [A_n] \cup \bigcup_{n\in\N} [B_n]) =: \tilde S,  
	\end{flalign*}
	
	\begin{figure}
		\centering
		\includegraphics[height=10cm]{circle-part-S.png}
		\caption{$S$ and the sets $A_n$ and $B_n$} \label{circleS}
	\end{figure}
	
	where $\bar S$ is the closure of $S$ (note that $\bar S = S\ \cup\ \{0\}$). Let $(\gamma,p)\in R$. Then $pe_\gamma\in \chi(\gamma,p)\cap S$ and therefore $\chi(\gamma,p)\in [\bar S]$. Assume that there exits an $n\in\N$ such that $\chi(\gamma,p)\cap A_n \neq \emptyset$. Then $\beta + \frac{\pi}{2} - \gamma < \frac{\pi}{2}$, hence $\beta < \gamma$, a contradiction. Similarly argument for any $B_n$, so we finally have $\chi(\gamma,p) \notin [A_n]$ and $\chi(\gamma,p) \notin [B_n]$ for any $n\in \N$. Hence $\chi(\gamma,p)\in\tilde S$ and therefore $\chi(R)\subset \tilde S$. \\
	\\
	Now let $(\gamma,p)\in\Phi$ such that $\chi(\gamma,p)\in \tilde S$. Assume that $\gamma\notin[\alpha,\beta]$ then with a similar argument as in the first inclusion it is easy to see that there must be an $n\in\N$ such that $\chi(\gamma,p)\cap A_n\neq \emptyset$ or $\chi(\gamma,p)\cap B_n\neq \emptyset$, a contradiction. Therefore $\gamma\in[\alpha,\beta]$. Now assume $p\notin (0,b]$. If $p>b$ then $\chi(\gamma,b)\cap B_b = \emptyset$ and since $\bar S\subset B_b$ it is $\chi(\gamma,p)\notin [\bar S]$, a contradiction. If $p<0$ then, since the angle between the segments $A_n$ and $B_n$ opposite of $S$ is strictly smaller than $\pi$, $\chi(\gamma,p)$ must intersect with $A_n$ or $B_n$ for some $n\in\N$, again a contradiction. Note that $p\neq 0$ since $[\{0\}] \subset [A_1]$. Thus we have $p\in(0,b]$. Therefore we have $(\gamma,p)\in R$ and finally $\tilde S\subset \chi(R)$. \\
	\\
	It is left to show that $\tilde S\in \mathcal{A}(2,1)$. All the segments $A_n$ and $B_n$ are compact and convex for all $n\in\N$, and since $S$ is bounded and convex as a circle segment with angle smaller than $\pi$, $\bar S$ is compact and convex. Finally $\tilde S\in \mathcal{A}(2,1)$. \\
	\\
	In total we get $\mathcal{R}_+\subset \mathcal{A}(2,1)$. $\mathcal{R}_-\subset \mathcal{A}(2,1)$ can be shown analogously. So for the last case let $R = [\alpha,\beta] \times \{0\}\in \mathcal{R}_0$. Define the line segment
	\begin{flalign*}
		T := \{(1-t)s_\alpha + ts_\beta\ |\ t\in [0,1]\}. 
	\end{flalign*}
	Then $[\{0\}] \cap [T] \in \mathcal{A}(2,1)$. We show that $\chi(R) = [\{0\}] \cap [T]$. Let $(\gamma,0)\in R$. Then $\chi(\gamma,0)$ contains $0$ and since $\gamma$ is in between $\alpha$ and $\beta$, $\chi(\gamma,0)$ must intersect with $T$ ($\autoref{T}$). For the other inclusion let $(\gamma,p)\in\Phi$ such that $\chi(\gamma,p)\in [\{0\}] \cap [T]$. Since $0\in \chi(\gamma,p)$ it must be $p=0$ and since it intersects with $T$ its angle must lay in between $\alpha$ and $\beta$. All in all this completes the proof. 
\end{proof}

\begin{figure}
	\centering
	\includegraphics[height=10cm]{T.png}
	\caption{Line in $[\{0\}] \cap [T]$} \label{T}
\end{figure}


\begin{definition}
	A $\mathcal{F}$-$\GG$-measurable function $g:\Omega \to \G$ is called a $\mathit{random\ line}$.  
\end{definition}

\begin{definition}
	We define the measure $\mu := {\lambda_2}_{|\Phi} \circ \chi^{-1}$ on $(\G,\GG)$ where ${\lambda_2}_{|\Phi}$ is the $2$-dimensional Lebesgue measure restricted to $\Phi$. We say a measure $\nu$ on $(\G,\GG)$ is locally finite if for any $K\in \K^2$ we have $\nu([K])<\infty$. 
\end{definition}

\begin{lemma}
	$\mu$ is locally finite and $G_2$-invariant. 
\end{lemma}
\begin{proof}
	Let $K\in \K^2$ and since $K$ is compact choose $r> 0$ such that $K\subset B_r$. Then we have $[K]\subset A_{B_r}$ and
	\begin{flalign*}
		\mu([K]) \leq \mu(A_{B_r}) = {\lambda_2}_{|\Phi} (\chi^{-1}(A_{B_r})) = {\lambda_2}_{|\Phi}([0,\pi)\times [-r,r]) = 2\pi r < \infty, 
	\end{flalign*}
	hence $\mu$ is locally finite. To show that $\mu$ is $G_2$-invariant, that means euclidean motion invariant, we must show it is translation and rotation invariant. First we clarify what exactly translation and rotation mean for lines. We denote $x\ modulo\ r$ as $(x)_r$. For $b\in\C$ and $\beta\in[0,2\pi)$ we define 
	\begin{flalign} \label{motion}
		T_b:\ &\Phi \to \Phi,\quad (\alpha,p) \mapsto (\alpha,p+\langle e_\alpha, b\rangle)
	\end{flalign}
	and
	\begin{flalign} \label{motion2}
		D_{\beta}:\ &\Phi \to \Phi,\quad (\alpha,p) \mapsto ((\alpha + \beta)_{\pi}, \delta((\alpha + \beta)_{2\pi})p), 
	\end{flalign}
	where 
	\begin{flalign*}
		\delta: [0,2\pi) \to \{-1,1\}, \quad \gamma \to \begin{cases}
			1,\ \gamma\in [0,\pi) \\
			-1,\ \gamma\in [\pi,2\pi)
		\end{cases}.
	\end{flalign*}
	It is easy to see that both functions all well-defined. $T_b$ defines a translation by $b$ and $D_\beta$ a rotation by $\beta$. Lets proof that first. Let $(\alpha,p)\in \Phi$, then 
	\begin{flalign*}
		x\in \chi(\alpha,p)+b &\Leftrightarrow x-b\in \chi(\alpha,p) \\ 
		&\Leftrightarrow \langle e_\alpha, x-b\rangle = p \\ 
		&\Leftrightarrow \langle e_\alpha, x\rangle = p + \langle e_\alpha, b\rangle \\
		&\Leftrightarrow x\in \chi(\alpha, p + \langle e_\alpha, b\rangle) \\
		&\Leftrightarrow x\in \chi(T_b(\alpha, p))
	\end{flalign*}
	and therefore $\chi(\alpha,p) + b = \chi(T_b(\alpha, p))$. Hence $T_b(\alpha,p)$ are indeed the parameters for the by $b$ translated line. For the rotation lets devide it into two cases. First let $(\alpha+\beta)_{2\pi} \in [0,\pi)$, then $\delta((\alpha+\beta)_{2\pi}) = 1$ and $(\alpha+\beta)_\pi = \alpha+\beta$ and therefore 
	\begin{flalign*}
		D_\beta(\alpha,p) = (\alpha+\beta,p).
	\end{flalign*} 
	In the second case with $(\alpha+\beta)_{2\pi} \in [\pi,2\pi)$ we have $\delta((\alpha+\beta)_{2\pi}) = -1$ and $(\alpha+\beta)_\pi = \alpha+\beta - \pi$ and therefore
	\begin{flalign*}
		D_\beta(\alpha,p) = (\alpha + \beta - \pi, -p).
	\end{flalign*}
	In the second case we have to carefully understand the parametrisation of $\G$, but finally we can see that $D_\beta(\alpha,p)$ are indeed the parameters of the by $\beta$ rotated line. \\
	\\We will further show now, that $\mu$ is invariant in respect to both these functions. Let $A\in \GG$, $b\in \C$ and $\nu_\beta\in SO_2$ for some $\beta\in[0,2\pi)$. We will understand $A+b = \{g+b\in \G\ |\ g\in A\}$ and $\nu_\beta A = \{\nu_\beta g\in \G\ |\ g\in A\}$ pointwise, and $g+b = \{x+b\ |\ x\in g\}$ and $\nu_\beta g=\{\nu_\beta x\ |\ x\in g\}$ pointwise aswell. We furthermore define $A_p := \{\alpha\in[0,\pi)\ |\ (\alpha,p)\in \chi^{-1}(A)\}$ and $A_\alpha := \{p\in \R\ |\ (\alpha,p)\in \chi^{-1}(A)\}$ for $(\alpha,p)\in \Phi$. For a translation we get 
	\begin{flalign*}
		\mu(A+b) 
		&= \int_\GG \1_{A+b}(g) \mu(dg) \\
		&= \int_\GG \1_A(g-b) \mu(dg) \\
		&= \int_\GG \1_A(g-b) {\lambda_2}_{|\Phi}(\chi^{-1}(dg)) \\
		&= \int_{\chi^{-1}(\GG)} \1_{\chi^{-1}(A)}(\chi^{-1}(g-b)) {\lambda_2}_{|\Phi}(d(\chi^{-1}(g))) \\
		&\overset{(\ref{motion})}{=} \int_\Phi \1_{\chi^{-1}(A)}(\alpha,p-\langle e_\alpha, b\rangle) {\lambda_2}_{|\Phi}(d(\alpha,p)) \\ 
		&= \int_0^\pi \int_\R \1_{A_\alpha}(p-\langle e_\alpha, b\rangle) {\lambda_1}(dp){{\lambda_1}_{|[0,\pi)}}(d\alpha) \\ 
		&= \int_0^\pi \int_\R \1_{A_\alpha +\langle e_\alpha, b\rangle}(p) {\lambda_1}(dp){{\lambda_1}_{|[0,\pi)}}(d\alpha) \\ 
		&= \int_0^\pi \lambda_1(A_\alpha +\langle e_\alpha, b\rangle) {{\lambda_1}_{|[0,\pi)}}(d\alpha) \\ 
		&\overset{(+)}= \int_0^\pi \lambda_1(A_\alpha) {{\lambda_1}_{|[0,\pi)}}(d\alpha) \\ 
		&= \dots \\
		&= \mu(A),
	\end{flalign*}
	and for a rotation we get
	\begin{flalign*}
		\mu(\nu_\beta A) &= \int_\G \1_{\nu_\beta A}(g) \mu(dg) \\
		&= \int_\G \1_A(\nu_{-\beta} g) {\lambda_2}_{|\Phi}(\chi^{-1}(dg))\\
		&= \int_{\chi^{-1}(\G)} \1_{\chi^{-1}(A)}(\chi^{-1}(\nu_{-\beta} g)) {\lambda_2}_{|\Phi}(d(\chi^{-1}(g)))\\
		&\overset{(\ref{motion2})}= \int_\Phi \1_{\chi^{-1}(A)}((\alpha - \beta)_{\pi}, \delta((\alpha - \beta)_{2\pi})p) {\lambda_2}_{|\Phi}(\alpha,p) \\
		&= \int_\Phi \1_{\chi^{-1}(A)}((\alpha - \beta)_{\pi}, \delta((\alpha - \beta)_{2\pi})p) {\lambda_2}_{|\Phi}(\alpha,p) \\
		&= \int_{0}^{2\pi} \int_\R \1_{\chi^{-1}(A)}((\alpha - \beta)_{\pi}, \delta((\alpha - \beta)_{2\pi})p) {\lambda_1}(dp){{\lambda_1}_{|[0,\pi)}}(d\alpha) \\
		&= \int_{0}^{2\pi} \int_\R \1_{A_{(\alpha - \beta)_{\pi}}}(\delta((\alpha - \beta)_{2\pi})p) {\lambda_1}(dp){{\lambda_1}_{|[0,\pi)}}(d\alpha) \\
		&\overset{(+)}= \int_{0}^{2\pi} \int_\R \1_{A_{(\alpha - \beta)_{\pi}}}(p) {\lambda_1}(dp){{\lambda_1}_{|[0,\pi)}}(d\alpha) \\
		&= \int_\R \int_{0}^{2\pi} \1_{\chi^{-1}(A)}((\alpha - \beta)_{\pi}, p) {{\lambda_1}_{|[0,\pi)}}(d\alpha){\lambda_1}(dp) \\
		&= \int_\R \int_{0}^{2\pi} \1_{A_p}((\alpha - \beta)_{\pi}) {{\lambda_1}_{|[0,\pi)}}(d\alpha){\lambda_1}(dp) \\
		&\overset{(+)}= \int_\R \int_{0}^{2\pi} \1_{A_p}(\alpha) {{\lambda_1}_{|[0,\pi)}}(d\alpha){\lambda_1}(dp) \\
		&= \int_\R \int_{0}^{2\pi} \1_{\chi^{-1}(A)}(\alpha,p) {{\lambda_1}_{|[0,\pi)}}(d\alpha){\lambda_1}(dp) \\
		&= \dots \\
		&= \mu(A).
	\end{flalign*}
	where in $(+)$ we used the translation and rotation invariance of the Lebesgue measure. This completes the proof.
\end{proof}

By \ref{uniqmeas} we know that $\mu$ is, up to a factor, the only euclidean motion invariant measure on $\G$. Since it is locally finite, for $K\in \K^2$ we can define a probability measure on $\G$ by
\begin{flalign*}
	\PP^K_\mu(A) := \frac{\mu( A\cap [K])}{\mu([K])},\quad A\in \GG.
\end{flalign*}

\begin{definition} \label{isotropic}
	Let $K\in\K^2$. A random line $g:\Omega \to \G$ is called $K$-$\mathit{isotropic}$ if 
	\begin{flalign*}
		\PP(g\in A) = \PP^K_\mu(A),\quad A\in\GG.
	\end{flalign*}
\end{definition}

\begin{lemma}\label{circ}
	Let $M,K\in \K^2$ with $M\subset K$. Let $f$ be a random $K$-isotropic and $g$ be a random $M$-isotropic line. Then for all $A\in \GG$ we have
	\begin{flalign*}
		\PP(f\in A\ |\ f\in [M]) = \PP(g\in A).
	\end{flalign*}
\end{lemma}
\begin{proof}
	Note that since $M\subset K$ it is $[M]\subset [K]$. For $A\in \GG$ we therefore directly get 
	\begin{flalign*}
		\PP(f\in A\ |\ f\in [M]) &= \frac{\PP(f\in A\cap [M])}{\PP(f\in [M])}\\
		&= \frac{\mu(A\cap [M]\cap [K])}{\mu([K])}\frac{\mu([K])}{\mu([M]\cap [K])}\\
		&=\frac{\mu(A\cap [M])}{\mu([M])} \\
		&= \PP(g\in A).
	\end{flalign*}
\end{proof}

If we choose a simple convex set such as $K=B_r$ the ball around the origin with radius $r$ then choosing random $K$-isotropic lines becomes a very intuitive and easy realizable task as the following lemma shows.

\begin{lemma} \label{chi}
	Let $K=B_r\in\K^2$ and let $(\alpha,p)$ be uniformly distributed in $\tilde \Phi:=[0,\pi)\times [-r,r]=\chi^{-1}([K])\subset \Phi$. Then $\chi(\alpha,p)$ is a random $K$-isotropic line. 
\end{lemma}
\begin{proof}
	For $A\in\GG$ we get 
	\begin{flalign*}
		\PP(\chi(\alpha,p)\in A) &= \PP((\alpha,p)\in \chi^{-1}(A)) = \frac{{\lambda_2}_{|\tilde\Phi}(\chi^{-1}(A)\cap \chi^{-1}([K])))}{{\lambda_2}_{|\tilde\Phi}(\chi^{-1}([K]))}\\
		&=\frac{{\lambda_2}_{|\Phi}(\chi^{-1}(A\cap [K]))}{{\lambda_2}_{|\Phi}(\chi^{-1}([K]))} = \frac{\mu(A\cap [K])}{\mu([K])}.
	\end{flalign*}
\end{proof}

Both lemmas \ref{circ} and \ref{chi} give a help for realizing $K$-isotropic lines for complicated sets $K$. Lemma \ref{circ} tells us that we if we are looking for a $K$-isotropic line, we can actually take a convex, compact set $B$ which contains $K$ and realize $B$-isotropic lines. If we realize such a line and it happens that it intersects $K$, we know that its distribution is equal to trying to realize $K$-isotropic lines directly. And how to realize $B$-isotropic lines? Lemma \ref{chi} tells us that if we choose $B=B_r$ a ball with a big enough radius such that it contains $K$, then realizing $B$-isotropic lines comes by choosing the line parameters $\alpha $ and $p$ uniformly in $[0,\pi)$ and $[-r,r]$. Finally we have a practicable process of choosing random $K$-isotropic lines, even if $K$ happens to be very asymmetric and complicated. This gives the base to define a new Incremental Aggregate in the next section which tries to approximize External DLA. 


\subsection{Definition}

We are now able to choose a line randomly out of all lines hitting a cluster in a fair and agreeable way, which are $K$-isotropic lines as presented in Definition \ref{isotropic}. All in all we have realized the mathematical structure to define the incremental aggregate as we have planned it in the motivation in the beginning of this chapter. Recall that for a bounded subset $A\subset \C$ the convex hull $conv(A)$ of $A$ is defined to be the smallest convex set containing $A$, formally 
\begin{flalign*}
	conv(A) := \bigcap_{A\subset K\in \K^2} K \in \K^2. 
\end{flalign*}
For a set $A\in \mathcal{P}_f$ we define 
\begin{flalign*}
	conv(A):=conv(\bigcup_{p\in A} sq(p))
\end{flalign*}
and since $\bigcup_{p\in A} sq(p)$ is a bounded set we have $conv(A)\in \K^2$. 

\begin{definition} $\mathit{(Random\ Line\ Hitting\ Distribution)}$ Let $A\in \mathcal{P}_f$ and $K := conv(A)\in \K^2$. For $x\in\Z^2$ and $g\in \G$ define
	\begin{flalign*} 
		\gamma_A(g) := \frac{1}{2}\1\{|g\cap A| \geq 2\} + \1\{|g\cap A| = 1\},
	\end{flalign*}
	\begin{flalign} \label{mu}
		\tilde \mu_A(x,g) := \gamma_A(g)
		\begin{cases}
			\1\{x\in \{\min(g\cap A),\max(g\cap A)\}\}, &\text{ if } g\cap A\neq \emptyset, \\
			0, &\text{ if } g\cap A = \emptyset.
		\end{cases}
	\end{flalign}
	and
	\begin{flalign}
		\mu_A(x) := \frac{1}{\PP^K_\mu([A])} \int_\G \ \tilde \mu_A(x,g) \ \PP^K_\mu(dg).
	\end{flalign}
	We quickly show that $\mu_A\in \mathcal{D}_A$. For all $x\in \Z^2\setminus A$ and $g\in \G$ we have $\tilde \mu_A(x,g) = 0$ and therefore $\mu_A(x) = 0$. Furthermore for all $g\in \G$ we have
	\begin{flalign*}
		\sum_{x\in A} \tilde \mu_A(x,g) = \begin{cases}
			\frac{1}{2}2, \quad &|g\cap A| \geq 2, \\
			1, \quad &|g\cap A| = 1, \\
			0, \quad &|g\cap A| = 0.
		\end{cases} \quad= \1\{g\cap A\neq \emptyset\}
	\end{flalign*} 
	and therefore 
	\begin{flalign*}
		\sum_{x\in A} \mu_A(x) = \frac{1}{\PP^K_\mu([A])}\ \int_\G \1\{g\cap A\neq \emptyset\}\ \PP^K_\mu(dg) = \frac{\PP^K_\mu([A])}{\PP^K_\mu([A])} = 1. 
	\end{flalign*}
	Hence, the family of distributions $(\mu_A)_{A\in \mathcal{P}_f}$ defines an incremental aggregate. 
\end{definition}

\begin{definition} $(\mathit{Line\ Hitting\ Aggregate)}$ An Incremental Aggregate with the Random Line Hitting Distribution we call $\mathit{Line\ Hitting\ Aggregate}$, short $\mathit{LHA}$. 
\end{definition}

\begin{remark}
	Note, that the indicator function in (\ref{mu}) is well-defined, since a line $g$ chosen by the distribution $\PP^K_\mu$ certainly hits $K=conv(A)$ and it is easy to understand that $g$ hits $conv(A)$ if and only if $g$ hits $A$. 
\end{remark}

\subsection{Fractal Dimension and Growth Rate of LHA}



\newpage
\section{Python Simulation}
Part of the work in this paper is the attempt to build simulations for both incremental aggregates $\mathit{External\ DLA}$ and $\mathit{LHI}$ which were presented in the previous sections. For calculating the simulations we use $\mathit{Python}$ and for rendering pictures we use the package $\mathit{pygame}$. Each vertex in $\Z^2$ is naturally mapped to its coordinate and is represented in this way in python code. In graphics each coordinate basically can be represented by squares in $\C$ as presented in Definition $\ref{squares}$, and each square will be represented by exactly one pixel, so every picture about incremental aggregates you'll see in the following consists of pixels each representing exactly one vertex in $\Z^2$ in the most natural way. Our aim will be to simulate the aggregates as close as possible to their mathematical definitions. This is much easier for LHI, and less obvious for External DLA. We start of with External DLA in the following.

\subsection{External DLA Simulation}

By identifying each vertex in $\Z^2$ with its square in $\C$ and a pixel when rendering, the representation of the space we are moving is exact. For any simulation of randomness we use the package $\mathit{random}$. Respecting the error behaviour of this package, random walks on $\Z^2$ can be simulated directly and therefore very exact. For External DLA we let a particle move randomly on the grid and wait for it to hit the actual cluster. The problem which will have to be solved is where to start the moving particle and how to handle a particle when it is moving away from the actual cluster, therefore creating a too long waiting time for it coming back to the cluster. In the definition of External DLA new particles start the random walk from infinity. Obviously this is not possible in simulation so this has to be solved differently. 



\subsection{LHA Simulation}


\newpage

\begin{thebibliography}{biblio}
\thispagestyle{empty}

\bibitem{stoch1}
Daniel Hug, Günter Last, Steffen Winter.
\emph{Stochastic Geometry, 	Lecture Notes (summer term 2020)}.
Institute of Technology, Karlsruhe, 2020

\bibitem{sackmann}
Franz Sackmann. 
\emph{Zufällige Geraden, Staatsexamensarbeit}.
University Karlsruhe (TH), Karlsruhe, 2007

\bibitem{lawler}
Gregory F. Lawler
\emph{Intersections of Random Walks}.
University of Chicago, 1996

\bibitem{henze}
Norbert Henze
\emph{Irrfahrten - Faszination der Random Walks}
Institute of Technology, Karlsruhe, 2018

\bibitem{fractalwinter}
Uta Freiberg, Ben Hambly, Michael Hinz and Steffen Winter
\emph{Fractal Geometry and Stochastics VI}
2020

\bibitem{magnetic}
J. Gonzalez-Gutierrez, J. L. Carrillo-Estrada and J. C. Ruiz-Suarez
\emph{Aggregation and dendritic growth in a magnetic granular system}
2013

\bibitem{hausdorff}
Egbert Brieskorn
\emph{Felix Hausdorff zum Gedächtnis, Band I, vieweg Verlag}
1996
page 160



\end{thebibliography}

\newpage
  
\thispagestyle{empty}

\vspace*{8cm}


\section*{Erklärung}

Hiermit versichere ich, dass ich diese Arbeit selbständig verfasst und keine anderen als die angegebenen Quellen und Hilfsmittel benutzt, die wörtlich oder inhaltlich übernommenen Stellen als solche kenntlich gemacht und die Satzung des Karlsruher Instituts für Technologie zur Sicherung guter wissenschaftlicher Praxis in der jeweils gültigen Fassung beachtet habe. \\[2ex] 

\noindent
Karlsruhe, den 10. März 2020\\[5ex] 

\end{document}

